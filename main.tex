\documentclass[12pt,a4paper,english]{article}

\sloppy

\usepackage[small]{complexity}
\usepackage[utf8]{inputenc}
\usepackage{cmap}
\usepackage[T2A]{fontenc}
\usepackage{csquotes}
\usepackage[english]{babel}
\usepackage{amssymb}
\usepackage{amsmath}
\usepackage{amsthm}
\usepackage{verbatim}
\usepackage[breaklinks]{hyperref}
\usepackage{algorithm}
\usepackage[noend]{algpseudocode}
\usepackage{comment}
\usepackage{bookmark}
\usepackage{mathtools}
\usepackage{makecell,xcolor}
\usepackage{tikz}
% \usepackage[backend=biber,style=alphabetic]{biblatex}
\usepackage{xspace}

\newtheorem{theorem}{Theorem}
\newtheorem{claim}{Claim}
\newtheorem{lemma}{Lemma}
\newtheorem{definition}{Definition}
\newtheorem{corollary}{Corollary}
\newtheorem{proposition}{Proposition}
\newtheorem{conjecture}{Conjecture}

\DeclarePairedDelimiter\set{\lbrace}{\rbrace}
\DeclarePairedDelimiter\abs{\lvert}{\rvert}
\DeclarePairedDelimiter\ceil{\lceil}{\rceil}
\DeclarePairedDelimiter\floor{\lfloor}{\rfloor}
\DeclarePairedDelimiter\segment{[}{]}

\newcommand{\map}[2]{\set{#1 \mid #2}}
\newcommand{\BigO}{\mathcal{O}}
\newcommand{\SmallO}{o}

\newcommand{\eps}{\varepsilon}

\newcommand{\Prob}[1]{P(#1)}
\newcommand{\Ex}[1]{\mathbb{E}\segment{#1}}

\begin{document}

In order to avoid problems with integers, we consider slightly modified version of the algorithm from the ``para''.

\begin{algorithm}
	\caption{}\label{ALG}
	\begin{algorithmic}[1]
		\Procedure{Init}{}
		\State Choose a random hash function $h: \segment{n} \to \segment{0, 1}$
		\Comment{Yes, we can't, but it is not important}
		\State $z \gets 1$
		\EndProcedure

		\Procedure{Process}{$j$}
		\State $z \gets \min(z, h(j))$
		\EndProcedure

		\Procedure{Output}{}
		\State \Return $\frac{1}{z} - 1$
		\EndProcedure
	\end{algorithmic}
\end{algorithm}

We will run $k$ (exact value will be defined later) copies of our algorithm and then output mean value of their answers. Let $\hat d_k$ denote the output of our algorithm and $d$ denote the true answer. We will also use $\hat d$ to denote $\hat d_1$ i.e. the answer of one copy of Algorithm~\ref{ALG}.

\begin{claim}\label{Main}
	For any $\eps$ there exists such $k$ that

	\begin{equation*}
		\Prob{\abs*{d - \hat d_k} \leq \eps d} \geq 1 - \delta
	\end{equation*}

	where $\delta$ is some constant less than $\frac12$.
\end{claim}

Let $X$ be a random variable uniformly distributed on $\segment{0, 1}$. Fix $d$. Then, $\hat d$ is equal to the minimum of $d$ i.i.d. copies of $X$. The cumulative distribution function for $\hat d$ is

\begin{equation*}
	F_{\hat d}(x) =
	\begin{cases}
		0,             & x < 0           \\
		1 - (1 - x)^d, & 0 \leq x \leq 1 \\
		1,             & x > 1
	\end{cases}
\end{equation*}

And, therefore, its density is

\begin{equation*}
	\rho_{\hat d}(x) =
	\begin{cases}
		d (1 - x)^{d - 1}, & 0 \leq x \leq 1  \\
		0,                 & \text{otherwise}
	\end{cases}
\end{equation*}

First, we need to compute $\Ex{\hat d}$.

\begin{equation*}
	\Ex{\hat d} = \int_0^1 \rho_{\hat d}(x) \cdot x = \int_0^1 d (1 - x)^{d - 1} x = \frac{1}{d + 1}
\end{equation*}

Note that it matches the formula in the output procedure of Algorithm~\ref{ALG}. Let $Y_d$ be $\hat d - \frac{1}{x + 1}$.

\begin{theorem}[Berry--Esseen]\label{BE}
	If $X_1, X_2, \ldots, $ are i.i.d. random variables such that

	\begin{itemize}
		\item $\Ex{X} = 0$

		\item $\Ex{X^2} = \sigma^2 > 0$

		\item $\Ex{\abs*{X}^3} = \rho < \infty$
	\end{itemize}

	and $F_n$ be the cumulative distribution function of $\frac{1}{\sigma \sqrt{n}} \sum_{i \leq n} X_i$, then, for all $x$ and $n$

	\begin{equation*}
		\abs*{F_n(x) - \Phi(x)} \leq \frac{C \rho}{\sigma^3 \sqrt{n}}
	\end{equation*}

	where $C$ is some positive constant and $\Phi(x)$ denotes the cumulative distribution function of the standard normal distribution.
\end{theorem}

Thus, we want to compute $\Ex{Y_d^2}$ and $\Ex{\abs*{Y_d}^3}$.

\begin{equation*}
	\Ex{Y_d^2} = \int_{0}^{1} d (1 - x)^{d - 1} \parens*{x - \frac{1}{d + 1}}^2 = \frac{d}{(1 + d)^2 (2 + d)}
\end{equation*}

Calculation of $\Ex{\abs*{Y_d}^3}$ can cause psychological trauma\footnote{Its true value is $\frac{2 d \parens*{ 6 d^{2 + d} + (1 + d)^d - d^2 (1 + d)^d }}{(1 + d)^{4 + d} (2 + d) (3 + d)}$. Check it if you are brave enough.}, so instead of doing that we bound $\rho$ with $\BigO \parens*{d^{-3}}$, see Appendix~\ref{App} for more. Thus, our constants are

\begin{align*}
	\sigma & = \frac{1}{d + 1}\sqrt{\frac{d}{d + 2}} \\
	\rho   & = \BigO \parens*{d^{-3}}
\end{align*}

Now it is time to recall our main goal.

By definition $\abs*{d - \hat d_k}$ is equal to arithmetic mean of $k$ i.i.d. copies of $Y_d$. Therefore, condition in Claim~\ref{Main} is equivalent to $F_k\parens*{\eps d \frac{\sqrt{k}}{\sigma}} - F_k\parens*{- \eps d \frac{\sqrt{k}}{\sigma}} \leq 1 - \delta$. After we apply Theorem~\ref{BE} we have to choose such $k$ that the following condition is satisfied.

\begin{equation}
	2 \Phi \parens*{ \eps d \frac{\sqrt{k}}{\sigma} } - 1 - 2\frac{C \rho}{\sigma^3 \sqrt{k}} \geq 1 - \delta
\end{equation}

Now it is obvious that such $k$ exists. Moreover,

\begin{align*}
	\frac{d}{\sigma}      & = d(d + 1)\sqrt{\frac{d + 2}{d}} = \Omega(d^2) \\
	\frac{\rho}{\sigma^3} & = \BigO(1)
\end{align*}

So, $k$ not only exists, but also does not depend on $d$. Now, it remains to see that once we found some suitable $k = k_0$ for some constant $\eps = \eps_0$, we can choose $k(\eps) = \parens*{\frac{\eps_0}{\eps}}^2 k_0$ for all $\eps \leq \eps_0$ and $k(\eps) = k_0$ for all $\eps > \eps_0$. Thus, $k = \BigO \parens*{\frac{1}{\eps^2}}$.

\appendix

\section{Appendix}\label{App}

We want to prove that $\Ex{\abs*{X}^3} = \BigO \parens*{d^{-3}}$. By definition,

\begin{equation*}
	\Ex{\abs*{X}^3} = \int_{0}^{1} d \parens*{1 - x}^{d - 1} \abs*{x - \frac{1}{d + 1}}^3
\end{equation*}

We split it into the following two parts

\begin{align}
	\int_{0}^{\frac{1}{d + 1}} d \parens*{1 - x}^{d - 1} \parens*{\frac{1}{d + 1} - x}^3 \label{XSmall} \\
	\int_{\frac{1}{d + 1}}^{1} d \parens*{1 - x}^{d - 1} \parens*{x - \frac{1}{d + 1}}^3 \label{XLarge}
\end{align}

First, it is easy to see that integral~\ref{XLarge} is the beta function.

\begin{equation*}
	\eqref{XLarge} = d \frac{d}{d + 1} \int_0^1 (1 - x)^{d - 1} x^3 = \frac{d^2}{d + 1} B(4, d)
\end{equation*}

Since $d$ is a positive integer, the factor $B(4, d)$ can be replaced with $\frac{3! \parens*{d - 1}!}{\parens*{d + 3}!}$ which is $\BigO\parens*{d^{-4}}$.

Thus, it remains to bound integral~\ref{XSmall}.

\begin{equation*}
	\eqref{XSmall} \leq d \int_{0}^{\frac{1}{d - 1}} \parens*{\frac{1}{d - 1} - x}^3 =  d \int_{0}^{\frac{1}{d - 1}} x^3 = \frac{d}{4} \parens*{\frac{1}{d - 1}}^4 = \BigO \parens*{d^{-3}}
\end{equation*}


\end{document}
